% !TEX root = ctfp-print.tex
\input{version}

\thispagestyle{empty}

\vspace*{80pt}

\begin{raggedleft}
\fontsize{24pt}{24pt}\selectfont
\textbf{프로그래머를 위한 \\ \trCategoryTheory}\\
\ifdefined\OPTCustomLanguage{%
  \vspace*{1cm}
  \small\selectfont{
    \textbf{\titlecap{\OPTCustomLanguage} Edition}\\
    \textit{Contains code snippets in Haskell and \titlecap{\OPTCustomLanguage}}\\
  }
}
\fi
\vspace*{1cm}
\fontsize{16pt}{18pt}\selectfont \textit{By } \textbf{Bartosz Milewski}\\
\vspace{1cm}
\fontsize{12pt}{14pt}\selectfont \textit{compiled and edited by}\\ \textbf{Igal Tabachnik}\\
\vspace{1cm}
\fontsize{12pt}{14pt}\selectfont \textit{translated by}\\ \textbf{jwvg0425, stet-stet}\\

\end{raggedleft}


\newpage

\vspace*{0.3\textheight}
\thispagestyle{empty}

\begin{small}
\begin{center}

\textsc{프로그래머를 위한 \trCategoryTheory}\\

\vspace{1.0em}
\noindent
Bartosz Milewski\\

\vspace{1.26em}
\noindent
Version \texttt{\OPTversion}\\\today


\vspace{1.6em}
\noindent
\includegraphics[width=3mm]{fig/icons/cc.pdf}
\includegraphics[width=3mm]{fig/icons/by.pdf}
\includegraphics[width=3mm]{fig/icons/sa.pdf}

\vspace{0.4em}
\noindent
이 작업물은 Creative Commons Attribution-ShareAlike 4.0 International\\
라이센스(\href{http://creativecommons.org/licenses/by-sa/4.0/}{\acronym{CC BY-SA 4.0}}) 아래에서 발행됩니다.

\vspace{1.26em}
\noindent
\href{https://bartoszmilewski.com/2014/10/28/category-theory-for-programmers-the-preface/}{Bartosz Milewski}의 블로그 포스트 시리즈로부터 제작되었습니다.\\
PDF와 책은 \href{https://hmemcpy.com}{Igal Tabachnik}에 의해 컴파일되었습니다.\\
\vspace{1.26em}
\noindent
\LaTeX{} 소스 코드는 깃헙에서 확인할 수 있습니다: \href{https://github.com/jwvg0425/milewski-ctfp-pdf-kr}{https://github.com/jwvg0425/milewski-ctfp-pdf-kr} 
\end{center}
\end{small}
