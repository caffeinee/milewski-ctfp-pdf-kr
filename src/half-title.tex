% !TEX root = ctfp-print.tex
\input{version}

\thispagestyle{empty}

\vspace*{80pt}

\begin{raggedleft}
\fontsize{24pt}{24pt}\selectfont
\textbf{프로그래머를 위한 \\ 카테고리 이론}\\
\ifdefined\OPTCustomLanguage{%
  \vspace*{1cm}
  \small\selectfont{
    \textbf{\titlecap{\OPTCustomLanguage} Edition}\\
    \textit{Contains code snippets in Haskell and \titlecap{\OPTCustomLanguage}}\\
  }
}
\fi
\vspace*{1cm}
\fontsize{16pt}{18pt}\selectfont \textit{By } \textbf{Bartosz Milewski}\\
\vspace{1cm}
\fontsize{12pt}{14pt}\selectfont \textit{compiled and edited by}\\ \textbf{Igal Tabachnik}\\

\end{raggedleft}


\newpage

\vspace*{0.3\textheight}
\thispagestyle{empty}

\begin{small}
\begin{center}

\textsc{Category Theory for Programmers}\\

\vspace{1.0em}
\noindent
Bartosz Milewski\\

\vspace{1.26em}
\noindent
Version \texttt{\OPTversion}\\\today


\vspace{1.6em}
\noindent
\includegraphics[width=3mm]{fig/icons/cc.pdf}
\includegraphics[width=3mm]{fig/icons/by.pdf}
\includegraphics[width=3mm]{fig/icons/sa.pdf}

\vspace{0.4em}
\noindent
This work is licensed under a Creative Commons\\
Attribution-ShareAlike 4.0 International License
(\href{http://creativecommons.org/licenses/by-sa/4.0/}{\acronym{CC BY-SA 4.0}}).

\vspace{1.26em}
\noindent
Converted from a series of blog posts by \href{https://bartoszmilewski.com/2014/10/28/category-theory-for-programmers-the-preface/}{Bartosz Milewski}.\\
PDF and book compiled by \href{https://hmemcpy.com}{Igal Tabachnik}.\\
\vspace{1.26em}
\noindent
\LaTeX{} source code is available on GitHub: \href{https://github.com/hmemcpy/milewski-ctfp-pdf}{https://github.com/hmemcpy/milewski-ctfp-pdf} 
\end{center}
\end{small}