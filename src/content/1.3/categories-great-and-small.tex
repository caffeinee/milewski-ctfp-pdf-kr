% !TEX root = ../../ctfp-print.tex

\lettrine[lhang=0.17]{다}{양한 예시들을} 살펴봄으로써 \trCategory에 대한 감을 잡을 수 있다.
매우 다양한 모양과 크기의 \trCategory\가 있는데, 생각치도 못한 곳에서 \trCategory\를 보게 되는 일도 있다.
간단한 예시부터 시작해보자.

\section{\trObject\가 없는 경우}

가장 간단한 \trCategory\은 0개의 \trObject, 따라서 0개의 \trMorphism\을 가진 \trCategory\이다. 
떼어놓고 본다면 굉장히 쓸쓸한 \trCategory지만, 다른 \trCategory\와 같이 본다면 중요할 수 있다.
이를테면, 모든 \trCategory의 \trCategory에서 그렇다 (그런 카테고리는 분명히 있다!)
공집합이 있어도 좋다면, 빈 \trCategory\가 있어도 괜찮지 않은가?

\section{단순그래프}

\trObject\를 \trArrow\로 잇는 것 만으로도 \trCategory\를 만들 수 있다.
임의의 유향그래프에 \trArrow\를 추가하여 \trCategory\를 만드는 상상을 해 보자.
먼저, 그래프의 각 \trNode에 \trIdentity  \trArrow\를 더한다. 
그 후, 한 \trArrow의 끝점이 다른 \trArrow의 시작점과 겹치는 각 경우에 대해
(즉, \trComposition\가능한 쌍에 대해) 두 \trArrow\를 \trComposition한 것을 추가한다.
새 \trArrow\를 추가할 때마다, 이 \trArrow\와 다른 \trArrow의 \trComposition에 대해서도 생각해야 한다.
(단, \trIdentity  \trArrow\는 고려하지 않는다.)
보통 무한히 많은 \trArrow\가 만들어지게 되는데, 문제는 없다.

위 과정을 다른 관점에서 이렇게도 볼 수 있다. 우리는 그래프의 각 \trNode마다 \trObject\를, 
\trComposition할 수 있는 \trEdge\을 이어놓은 것 하나마다 \trMorphism\를 가지는 \trCategory\를 만들었다.
(\trIdentityMorphism\은 특별히 0개의 \trEdge\을 이어놓은 것으로 생각할 수도 있다.)

그러한 \trCategory\는 주어진 그래프로부터 생성된 \trFreeCategory\라고 부른다.
이는 \trCategory의 \trFreeConstruction예시이기도 하다. \trFreeConstruction이란, 
미완성인 구조가 주어지면 이를 확장하여, 그 구조의 성질을 만족시키면서 최소의 내용물을 포함하게 하는 것을 말한다. %차후검토1
여기서는 \trCategory의 성질을 만족하게끔 하였다. 
앞으로 \trFreeConstruction의 예시를 많이 보게 될 것이다.

\section{\trOrder}

이제 완전히 다른 예시를 하나 살펴보자.
\trMorphism\이 \trObject 사이의 관계를 나타내는 \trCategory, 특히 ``같거나 작다''는 관계를 나타내는 \trCategory\를 살펴보자.
먼저 이것이 정말 카테고리인지를 살펴보자. 
\trIdentityMorphism\은 있는가? 모든 \trObject\는 자기 자신보다 같거나 작으므로, 그렇다!
\trComposition\은 있는가? $a \leqslant b$ 이고 $b \leqslant c$ 이면 $a \leqslant c$이므로, 그렇다!
\trComposition\은 \trAssociativity\를 만족하는가? 그렇다!
위와 같은 ``관계''를 가지는 집합은 \newterm{\trPreorder}라고 부르는데, 위와 같은 이유로 \trPreorder\는 \trCategory이다.

다른 더 강한 관계를 사용할 수도 있다. 이를테면, 위 조건에 $a \leqslant b$인 동시에 $b \leqslant a$이면 $a$는 $b$와 같아야 한다는 조건을 추가할 수 있다.
이를 만족하는 집합을 \newterm{\trPartialOrder}라고 한다.

마지막으로, 임의의 두 \trObject사이에 관계가 있다는 조건을 둘 수 있다.
이 조건을 만족하는 집합을 \newterm{\trLinearOrder} 또는 \newterm{\trTotalOrder}라고 부른다.

위 순서가 있는 집합들을 \trCategory\로 보고 성질을 찾아보자.
\trPreorder\는 임의의 \trObject $a$와 $b$사이에 최대 1개의 \trMorphism\이 존재하는 \trCategory다.
이러한 성질을 만족하는 \trCategory\를 ``\trThin''다고(thin) 부른다. \trPreorder\는 \trThin\은 \trCategory다.


\trCategory  $\cat{C}$ 내의 \trObject  $a$ 에서 \trObject  $b$로 가는 \trMorphism의 집합은
\newterm{\trHomSet}(hom-set)이라고 부르며, $\cat{C}(a, b)$ (또는 $\mathbf{Hom}_{\cat{C}}(a, b)$)라고 쓴다. 
즉, 임의의 \trPreorder에 속한 모든 \trHomSet\은 공집합이거나, 하나의 원소만을 가진다. 
$a$에서 $a$로 가는 \trMorphism의 집합인 $\cat{C}(a, a)$도 예외는 아니며, 이 집합은 어떤 \trPreorder에서든 항등 함수 하나만을 포함한다.
한편, \trPreorder에는 사이클이 있을 수 있지만, \trPartialOrder에는 있을 수 없다.

\trPreorder, \trPartialOrder, \trTotalOrder\를 제대로 알아보는 것은 중요하다. 정렬 때문이다.
버블 소트, 퀵 소트, 머지 소트 등의 정렬 알고리즘은 \trTotalOrder상에서만 제대로 작동한다.
\trPartialOrder\는 위상 정렬을 사용하면 정렬할 수 있다.

\section{집합으로서의 \trMonoid}

\trMonoid\는 부끄러울 정도로 간단하지만 놀라울 정도로 유용한 개념이다.
기본적인 사칙연산의 이면에 숨겨진 개념이기도 하다. 덧셈, 곱셈이 각각 \trMonoid\를 이루기 때문이다.
\trMonoid\는 프로그래밍에서는 어디서든 찾아볼 수 있다.
문자열, 리스트, ``접을 수 있는(foldable)''\footnote{역주. 이 용어에 대한 설명은 이후에 나올 것이다.} %차후검토2 
자료구조, 동시성 프로그래밍에서 사용하는 퓨쳐(future), 함수형 리액티브 프로그래밍에서의 이벤트(events) 
등의 형태로 \trMonoid\를 찾아볼 수 있다.

전통적으로는, \trMonoid\는 이항 연산이 있는 집합으로 정의된다. 이 연산은 \trAssociativity\를 만족하며, 
이 연산에 대해 항등원과 비슷하게 행동하는 특별한 원소가 하나 있으면 된다.

예를 들어, 0 이상의 정수로 구성된 집합은 덧셈에 대해 \trMonoid\를 이룬다.

\trAssociativity란 다음을 의미한다.
\[(a + b) + c = a + (b + c)\]
즉, 수를 더할 때 괄호를 생략해도 괜찮다는 것이다.

\trNeutralElement\는 0이다. 왜냐하면,
\[0 + a = a\]
이고
\[a + 0 = a\]
이기 때문이다. 여기서 두 번째 식은 첫 번째 식과 같은 말인데, 덧셈이 \trCommutativity\를 만족하기 때문이다 ($(a + b
= b + a)$).
그러나, \trCommutativity\은 \trMonoid의 정의에 포함되어 있지 않다.
예를 들어서, 문자열 연결(concatenation) 연산은 \trCommutativity\를 만족하지 않음에도 \trMonoid\를 이룬다.
참고로 문자열 연결 연산에 대한 \trNeutralElement\는 빈 문자열인데, 빈 문자열은 임의의 문자열의 양쪽에 붙어도 그것을 바꾸지 않는다.

하스켈에서는 \code{mempty}라는 \trNeutralElement\와 \code{mappend}라는 이항연산이 있는 \trMonoid 타입 클래스를 다음과 같이 선언할 수 있다. 

\src{snippet01}
두 개의 \trArgument\를 받는 함수의 표현인
\code{m -> m -> m}는 처음에는 이상해 보일 수 있으나,
\trCurrying(currying)에 관한 얘기를 하고 나면 완벽하게 말이 됨을 볼 수 있을 것이다. 
여러 개의 \trArrow\를 가진 표현은 기본적으로 두 가지로 해석될 수 있다. 첫째는 다\trArgument 함수, 
둘째는 하나의 \trArgument\를 받아서 함수를 리턴하는 함수이다. 
두 번째 해석은 \code{m -> (m -> m)}와 같이 괄호를 써 주면 더 잘 이해할 수 있으나, 
애초에 화살표 기호가 오른쪽부터 \trAssociate하므로 괄호가 필요가 없다.
이 해석에 대해서는 잠시 후에 다시 돌아와 이야기하도록 한다.

하스켈에서는 \code{mempty}와 \code{mappend}의 \trMonoid 관련 성질을 나타낼 방법이 없다.
즉, \code{mempty}가 \trNeutralElement이고 \code{mappend}가 \trAssociativity\을 만족함을 나타낼 수 없다.
그러한 성질이 만족되게끔 하는 것은 프로그래머의 책임이다.

하스켈의 클래스는 C++의 클래스처럼 문법이 강제적이지 않다.
새로운 타입을 정의할 때 맨 처음에 클래스를 지정해줄 필요가 없다. 
주어진 타입이 다른 클래스의 인스턴스라는 점을 훨씬 나중에 선언해도 된다.
예를 들어, \code{mempty}와 \code{mappend}의 구현을 제공해서 \code{String}를 \trMonoid\로 선언해보자.
(사실 Prelude에 이미 되어 있다.)

\src{snippet02}
\code{String}은 문자열일 뿐이므로, 여기서 우리는 리스트 연결 연산자 \code{(++)}를 사용했다.

하스켈의 문법에 대해 한 마디 하고 지나가는 게 좋을 듯 하다. 임의의 중위연산자를 괄호로 싸면 그것은 두 개의 \trArgument\를 취하는 함수가 된다.
두 문자열이 주어지면, \code{++}를 중간에 넣어서 둘을 이어줄 수 있다.

\begin{snip}{haskell}
"Hello " ++ "world!"
\end{snip}
또는, 괄호로 싸서 만든 \code{(++)}의 두 \trArgument\로 제공해도 된다.

\begin{snip}{haskell}
(++) "Hello " "world!"
\end{snip}
함수에 전달하는 \trArgument 사이에 쉼표를 쓰거나 \trArgument\를 괄호로 감싸지 않는다. 
아마 이 점이 하스켈을 배우면서 가장 적응하기 어려운 부분 중 하나일 것이다.

하스켈에서는 함수의 ``같음''을 표현할 수 있다는 점도 짚고 넘어가면 좋을 듯 하다. 이를테면 다음과 같다.

\begin{snip}{haskell}
mappend = (++)
\end{snip}
관념적으로, 이는 함수가 만들어내는 값의 ``같음''을 표현하는 것과는 다르다.

\begin{snip}{haskell}
mappend s1 s2 = (++) s1 s2
\end{snip}
전자는 \trCategory $\Hask$(혹은 \trBottom(끝나지 않는 계산을 뜻함)\을 무시하면 $\Set$)에서의 \trMorphism의 동등성으로 해석할 수 있다.
이러한 식은 더 간결할 뿐 아니라, 다른 \trCategory\로 일반화시킬 수 있다.
후자는 \newterm{\trExtensional} 동등성이라고 부르며,
어떤 두 입력 문자열에 대해 \code{mappend}와 \code{(++)}가 같음을 의미한다.
\trArgument의 값이 가끔 \newterm{점}이라고 불리기도 하므로 (점 $x$에서 함수 $f$의 값), 이러한 표현은 
\trPointWiseEquality\라고 부른다.
\trArgument\를 지정하지 않는 함수의 동등성은 \newterm{\trPointFree}하다고 부른다.
(우연히도, \trPointFree 식은 함수의 \trComposition\과 자주 관계되곤 하는데, \trComposition\은 점을 이용해 표현하므로, 초보자에겐 조금 헷갈릴 수도 있다.)

C++에서 \trMonoid 선언과 가장 비슷한 것은 아마 다음 표준에 제안된 문법인 콘셉트(concepts)이다.

\begin{snip}{cpp}
template<class T>
  T mempty = delete;
  
template<class T>
  T mappend(T, T) = delete;
  
template<class M> 
  concept bool Monoid = requires (M m) { 
    { mempty<M> } -> M; 
    { mappend(m, m); } -> M;
  };
\end{snip}
첫 선언은 값 템플릿을 사용하는데, 이 문법도 다음 표준에 제안되어 있는 상황이다.
\trPolymorphic 값이란 각 타입마다 다른 값을 가지는, 값의 집단이다.

\code{delete}라는 키워드는 디폴트 값이 정의되어있지 않음을 말한다. 
즉, 값이 경우별로 따로 지정되어야 한다.
비슷하게, \code{mappend}도 디폴트값이 없다.

\code{Monoid}라는 콘셉트는
주어진 타입 \code{M}에 대해 \code{mempty}와 \code{mappend}에 대한 적절한 정의가 존재하는지 검증하는 \trPredicates다.
\code{bool} 타입인 것도 이 때문이다.

적절한 오버로드와 템플릿 특수화를 제공하여서 \trMonoid 콘셉트\를 인스턴스화할 수 있다.

\begin{snip}{cpp}
template<>
std::string mempty<std::string> = {""};

std::string mappend(std::string s1, std::string s2) { 
    return s1 + s2;
}
\end{snip}

\section{\trCategory\로서의 \trMonoid }

지금까지는 집합의 원소\를 가지고 ``익숙한'' 방식으로 \trMonoid\를 정의해보았다. 
그러나, \trCategoryTheory에서는 집합과 그 원소가 아니라, \trObject\와 \trMorphism에 대한 이야기를 한다.
그러므로 우리는 관점을 살짝 바꾸어서, 집합 안에서 여러 가지를 ``움직이''거나 ``미는'' 이항연산자를 생각해보자.

이를테면, 모든 자연수에 5를 더하는 연산을 생각해볼 수 있다. 이 연산을 통해 0은 5로, 1은 6으로, 2는 7로 대응된다.
이는 자연수집합에서 정의된 함수과 같다. 좋다. 이제 함수와 집합이 있다. 
일반적으로, 어떤 수 $n$에 대해서도 $n$을 더하는 함수가 있다. 이 함수를 $n$의 ``\trAdder''\라고 부른다.

\trAdder\는 어떻게 \trComposition할 수 있을까? 5를 더하는 함수와 7을 더하는 함수의 \trComposition\은 12를 더하는 함수다.
즉 \trAdder의 \trComposition\과 덧셈에 적용되는 법칙은 동치나 다름없다. 역시 좋다. 이제 덧셈을 함수의 \trComposition\로 바꾸어 볼 수 있다.

그러나 이게 끝이 아니다. \trNeutralElement인 0의 \trAdder도 있다. 0을 더한다고 해서 움직여지는 건 아무 것도 없으므로,
이 \trAdder\은 자연수 집합에서의 \trIdentity 함수이다.

보통의 덧셈 법칙을 나열하는 대신, \trAdder의 법칙을 나열해도 손실되는 정보는 없을 것이다.
함수의 \trComposition\이 \trAssociativity\를 만족하므로 \trAdder의 \trComposition도 \trAssociativity\을 만족한다는 점을 관찰할 수 있다.
또한, \trIdentity 함수에 해당하는 영\trAdder도 있음을 관찰할 수 있다.

일부 머리 좋은 독자는 정수에서 \trAdder\로의 사상이, 사실은 \code{mappend}의 타입에 대한 표현인 \code{m -> (m -> m)}로부터 유도될 수 있음을 눈치챘을 것이다.
이를 통해서 \code{mappend}가 \trMonoid 집합의 원소를, 그 집합에 작용하는 함수로 사상한다는 점을 알 수 있다.

이제, 지금껏 자연수 집합에서 여러 가지를 해왔다는 사실을 잊어버리고, 대신 이 집합을 \trMorphism\이(\trAdder\가) 많이 달린 \trObject\라고 생각하자.
\trMonoid(monoid)는 단 한 개의 \trObject\를 가진 \trCategory다. 사실 \trMonoid(monoid)라는 단어도 ``하나''를 뜻하는 그리스어 \emph{mono}에서 온 것이다.
모든 \trMonoid\는 단 한 개의 \trObject\와, 적절하게 \trComposition에 관련된 법칙을 따르는 \trMorphism의 집합으로 설명될 수 있다.

\begin{figure}[H]
\centering
\includegraphics[width=0.35\textwidth]{images/monoid.jpg}
\end{figure}

\noindent
재미있는 케이스로 문자열 연결 연산이 있는데, 오른쪽에 연결할 수도 있고, 왼쪽에 연결(전연결)할 수도 있기 때문이다.
이 두 경우 \trComposition 관련 식이 서로 대칭적이다.
``bar'' 뒤에 ``foo''를 붙인 결과와 ``foo'' 앞에 ``bar''를 붙인 결과가 같음을 생각하면 쉽게 납득이 갈 것이다.

이런 질문을 할 수도 있다. 임의의 \trCategory적 \trMonoid(\trObject\가 하나인 \trCategory)\로부터, 집합과 이항연산자를 포함하는 \trMonoid\를 유일하게 정의할 수 있을까?
\trObject\가 하나인 \trCategory에서는 항상 집합을 추출해낼 수 있음이 증명되어 있다. 이 집합은 \trMorphism의 집합이고, 위에서 든 예시에서는 \trAdder였다.
달리 말하면, \trCategory $\cat{M}$ 내의 유일한 \trObject $m$에 대한 \trHomSet $\cat{M}(m, m)$이 있다. 
우리는 이 집합에서 이항연산자를 쉽게 정의해낼 수 있다. 집합에 속하는 두 원소의 "모노이드곱"은 각각에 대응되는 \trMorphism의 \trComposition\이다.
$\cat{M}(m, m)$의 원소 중 $f$와 $g$에 각각 대응되는 두 원소의 곱은 그 \trComposition인 $f \circ g$이다.
이 \trMorphism들의 시작점과 끝 점이 항상 같은 \trObject이므로, \trComposition\은 항상 존재한다. %어라? starting point는 용어가 뭐였더라요...?
또한 \trCategory의 성질에 의해서 이 곱연산은 \trAssociativity\를 만족한다. \trIdentityMorphism\이 이 곱연산의 \trNeutralElement다. 
따라서 우리는 \trCategory\trMonoid\로부터 집합 \trMonoid\를 만들어낼 수 있다. 어떻게 보든 이 두 가지 \trMonoid\는 같은 것이다.

\begin{figure}[H]
\centering
\includegraphics[width=0.4\textwidth]{images/monoidhomset.jpg}
\caption{집합 내의 점과 \trMorphism\로 보는 \trMonoid\trHomSet}
\end{figure}

\noindent
수학자들이 지적할 수 있는 사소한 문제가 하나 있다. \trMorphism이 집합을 이루리라는 보장이 없다는 것이다.
\trCategory의 세계에서는, 집합보다 큰 것들이 있다. 임의의 두 \trObject 사이에 존재하는 \trMorphism\이 집합을 이루는 \trCategory\는 \trLocallySmall다고 부른다.
약속한 대로 이런 미묘한 부분은 무시하고 넘어갈 것이나, 혹시 모르니 일단 언급해둔다.

\trCategoryTheory에서 여러 흥미로운 현상의 근본에는 \trHomSet의 원소를 \trComposition의 규칙을 따르는 \trMorphism\으로도, 집합의 점으로도 볼 수 있다는 사실이 있다.
$\cat{M}$ 에 포함된 \trMorphism의 \trComposition\은 집합 $\cat{M}(m, m)$에서의 모노이드곱으로 변환될 수 있다.

\section{연습문제}

\begin{enumerate}
\tightlist
\item
  다음으로부터 \trFreeCategory\를 만들어내시오.  

  \begin{enumerate}
  \tightlist
  \item
    \trEdge 없이 \trNode 하나만 가지는 그래프
  \item
    하나의 (유향) \trEdge, 그리고 \trNode 하나를 가지는 그래프.
    (힌트: 이 \trEdge\은 자기 자신과 \trComposition될 수 있다.)
  \item
    두 개의 \trNode\과 그 사이를 잇는 \trArrow\를 가지는 그래프.
  \item
    한 개의 \trNode\과, 알파벳 a, b, c \ldots{} z로 표시된 26개의 \trArrow\로 이루어진 그래프
  \end{enumerate}
\item
  다음은 어떤 \trOrder인가?

  \begin{enumerate}
  \tightlist
  \item
    포함 관계를 만족하는 집합의 집합. $A$의 원소가 모두 $B$의 원소라면 $A$가 $B$에 `포함된다'고 한다. 
  \item
    C++ 타입. C++에서는 \code{T2}형 포인터를 넘겨야 한다고 지정된 곳에 \code{T1}형 포인터를 컴파일 에러 없이 넘길 수 있다면 
    \code{T1}은 \code{T2}의 서브클래스이다.
  \end{enumerate}
\item
  \code{Bool}은 \code{\&\&} (AND) 연산자와 \code{||} (OR) 연산자에 대해서 각각 하나씩의 (집합론적인) \trMonoid\를 이룸을 보이시오.
  \code{Bool}이 \code{True}와 \code{False}으로 이루어진 집합임을 이용하시오.
\item
  AND 연산자에 대한 \code{Bool} \trMonoid\를 \trCategory\로 표현하시오. 모든 \trMorphism, 각각의 \trComposition 법칙을 나열하시오.
\item
  법 3(modulo 3)에 대한 덧셈을 \trMonoid \trCategory\로 표현하시오\footnote{역주. 3으로 나눈 나머지에 대한 덧셈을 말한다. 2+2가 1이 되는(정확히는 ``합동인'') 식이다. }. 
\end{enumerate}