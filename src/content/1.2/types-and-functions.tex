% !TEX root = ../../ctfp-print.tex

함수와 타입 \trCategory\는  프로그래밍에서 중요한 역할을 담당한다. 그러니 타입이 무엇이고 언제 그것이 필요한지 얘기해보자.

\section{타입이 누구에게 필요한가?}

정적 타입 vs 동적 타입, 강 타입 vs 약 타입 각각의 이점에 관해 약간의 논란이 있는 것 같다.
각각의 선택을 사고 실험을 통해 표현해보자. 수백만 마리의 원숭이가 컴퓨터 키보드 앞에서 아무 키나 랜덤하게 누르고, 프로그램을 작성하고, 컴파일하고, 실행하고 있다.

\begin{figure}[H]
\centering
\includegraphics[width=0.3\textwidth]{images/img_1329.jpg}
\end{figure}

\noindent
기계어의 경우, 원숭이들이 만든 어떤 종류의 바이트열 조합도 허용되며 실행될 수 있다.
하지만 좀 더 고수준 언어의 경우, 컴파일러가 어휘적(lexical), 문법적 오류를 잡아낼 수 있다는 사실에 감사할 수 있을 것이다.
많은 원숭이들이 바나나 없이 떠나게 되겠지만, 남은 프로그램들은 유용해질 수 있는 좀 더 나은 기회를 갖게 될 것이다.
타입 검사는 무의미한 프로그램에 대한 또다른 방어책을 제공해준다.
더 나아가, 타입 불일치를 실행 중에만 감지할 수 있는 동적 타입 언어와 다르게 정적 타입 언어는 이를 컴파일 타임에 잡아낼 수 있게 해 준다.
실행하기 전에 수많은 잘못된 프로그램을 잡아낼 기회를 제공하는 것이다.

그래서 질문은 이거다. 우리는 원숭이를 행복하게 하기를 바라는 걸까, 아니면 올바른 프로그램을 만들어내길 원하는 걸까?

타이핑하는 원숭이 사고실험의 일반적인 목표는 셰익스피어 작품을 완벽하게 만들어내는 것이다.
이 반복에서 맞춤법 검사기와 문법 검사기를 갖게 되는 것은 성공 가능성을 엄청나게 증가시킨다.
타입 검사기와 비슷한 것이 있다면, 로미오가 사람으로 선언될 경우 그에게서 나뭇잎이 자라나거나, 그의 강력한 중력장이 광자(photon)를 붙잡아두게 된다거나 하지 못하게 만듦으로써 더욱 멀리까지 나아갈 수 있을 것이다.

\section{타입은 \trComposability에 관한 것}

\trCategoryTheory이란 곧 \trArrow\를 합성하는 것이다. 하지만 어느 두 \trArrow도 합성될 수 없다. 
한 \trArrow의 \trTargetObject\는 반드시 다음 \trArrow의 \trSourceObject\와 같아야만 한다.
프로그래밍에서 우리는 한 함수의 결과를 다른 함수로 전달할 수 있다.
만약 \trTargetFunction\이 \trSourceFunction에서 만든 데이터를 올바르게 해석할 수 없다면 프로그램은 동작하지 않을 것이다.
이 두 부분은 \trComposition\이 동작하기 위해 반드시 일치해야 한다.
언어의 타입 시스템이 강하면 강할 수록 이 두 가지가 일치하는지 여부를 더 잘 표현하고 기계적으로 판별 가능하게 된다.

강 타입에 대해 내가 들은 이야기 중 유일하게 일리가 있었던 주장은 강 타입이 의미론적으로는 올바른 몇몇 프로그램을 거부할 수 있다는 것이었다.
하지만 실제로 이런 일은 굉장히 드물게 일어나며, 어느 경우에서든 모든 언어들은 정말 필요할 경우 타입 시스템을 우회할 수 있도록 만들어주는 백도어를 제공한다.
심지어 하스켈 마저 \code{unsafeCoerce}를 제공한다. 하지만 이런 장치들은 반드시 사려깊게 사용되어야 한다.
프란츠 카프카의 소설 변신의 주인공 그레고르는 거대한 벌레로 변신하면서 타입 시스템을 부숴버렸고, 그 결과가 어땠는지는 모두들 잘 알고 있을 것이다.

흔히 듣는 또다른 주장은 타입을 다루는 일이 프로그래머에게 너무 큰 부담을 지운다는 것이었다. 
C++에서 반복자 선언을 몇 번 하고 나면 느낄 이런 감정적인 부분에 공감은 간다.
하지만, 사용된 문맥에 따라 컴파일러가 자동으로 대부분의 타입을 추정해내는 \newterm{\trTypeInference}으로 불리는 기술이 존재한다.
C++에서 \code{auto}를 이용해 변수를 선언하면 컴파일러가 자동으로 그 변수의 타입을 지정해줄 것이다.

하스켈에서는 아주 희귀한 몇몇 경우를 제외하고는 \trTypeAnnotation\은 선택사항에 불과하다.
하지만 그 사실과는 별개로 프로그래머들은 \trTypeAnnotation\을 사용하려고 하는 경향이 있다.
왜냐하면 타입 \trAnnotation\은 코드의 의미에 대해 굉장히 많은 것을 말해주고, 컴파일 에러를 더 쉽게 이해할 수 있게 만들어주기 때문이다.
하스켈에서 타입 설계로부터 프로젝트를 시작하는 것은 흔히 쓰이는 방법이다. 
\sloppy{Later, type annotations drive the implementation
and become compiler-enforced comments.}

강한 정적 타입은 코드를 테스트하지 않는 것에 대한 변명거리로 자주 사용된다.
당신은 하스켈 프로그래머가 ``컴파일이 된다면, 그건 제대로 동작하는 프로그램일 것이다''라고 말하는 것을 몇 번 들은 적이 있을 지도 모른다.
물론, 타입이 올바르다고 해서 제대로된 동작을 할 것이라는 보장은 어디에도 없다.
이런 무신경한 태도는 여러 연구에서 예상과 달리 하스켈이 코드 퀄리티 측면에서 크게 앞서있지 않다는 결과로 나타난다.
It seems that, in the commercial setting, the pressure to
fix bugs is applied only up to a certain quality level, which has
everything to do with the economics of software development and the
tolerance of the end user, and very little to do with the programming
language or methodology. A better criterion would be to measure how many
projects fall behind schedule or are delivered with drastically reduced
functionality.

유닛 테스트가 강타입을 대체할 수 있다는 주장의 경우, "특정 함수가 받는 인자의 타입을 바꾸기"라는 강 타입 언어에서 흔한 리팩토링 과정을 생각해보자.
강 타입 언어의 경우 단순히 함수의 선언을 바꾸고 그 다음 일어나는 모든 빌드 에러들을 수정하는 것으로 충분하다.
반면 약 타입 언어에서는 타입을 바꿔도 그 함수를 호출할 때 해당 함수가 타입과 안 맞는 데이터가 전달되지 않을 것이라 가정한다는 사실만이 남는다.
유닛 테스트가 타입 불일치중 일부를 잡아줄지도 모르지만, 테스트는 항상 결정적이기보다는 확률적인 과정이다. 테스트는 증명의 형편없는 대체제일 뿐이다.

\section{타입이란 무엇인가?}

타입에 대한 가장 간단한 직관은 이것을 값의 집합으로 보는 것이다.
\code{Bool}이라는 타입(하스켈에서 타입은 대문자로 시작한다는 사실을 기억하자)은 \code{True}와 \code{False}라는 두 개의 값으로 이루어진 집합이다.
\code{Char}이라는 타입은 \code{a}나 \code{ą} 등의 유니코드 문자로 이루어진 집합이다.

집합은 유한할 수도 있고 무한할 수도 있다. \code{Char}의 리스트와 동의어인 \code{String} 타입을 무한 집합의 예시로 들 수 있다.

\code{x}를 \code{Integer} 타입으로 선언해보자.

\src{snippet01}
우리는 이제 이 값이 정수 집합의 한 원소라고 얘기할 수 있다.
하스켈에서 \code{Integer}는 무한 집합이며, 임의 크기의 산술 연산을 하는데 사용될 수 있다. \code{Int}라는 유한한 크기의 집합도 있는데 이는 C++의 \code{int}처럼 동작하는 기기에 부합하는 크기를 가진다.


There are some subtleties that make this identification of types and
sets tricky. There are problems with polymorphic functions that involve
circular definitions, and with the fact that you can't have a set of all
sets; but as I promised, I won't be a stickler for math. The great thing
is that there is a category of sets, which is called $\Set$, and
we'll just work with it. In $\Set$, objects are sets and morphisms
(arrows) are functions.

$\Set$ is a very special category, because we can actually peek
inside its objects and get a lot of intuitions from doing that. For
instance, we know that an empty set has no elements. We know that there
are special one-element sets. We know that functions map elements of one
set to elements of another set. They can map two elements to one, but
not one element to two. We know that an identity function maps each
element of a set to itself, and so on. The plan is to gradually forget
all this information and instead express all those notions in purely
categorical terms, that is in terms of objects and arrows.

In the ideal world we would just say that Haskell types are sets and
Haskell functions are mathematical functions between sets. There is just
one little problem: A mathematical function does not execute any code
--- it just knows the answer. A Haskell function has to calculate the
answer. It's not a problem if the answer can be obtained in a finite
number of steps --- however big that number might be. But there are some
calculations that involve recursion, and those might never terminate. We
can't just ban non-terminating functions from Haskell because
distinguishing between terminating and non-terminating functions is
undecidable --- the famous halting problem. That's why computer
scientists came up with a brilliant idea, or a major hack, depending on
your point of view, to extend every type by one more special value
called the \newterm{bottom} and denoted by \code{\_|\_}, or
Unicode $\bot$. This ``value'' corresponds to a non-terminating computation.
So a function declared as:

\src{snippet02}
may return \code{True}, \code{False}, or \code{\_|\_};
the latter meaning that it would never terminate.

Interestingly, once you accept the bottom as part of the type system, it
is convenient to treat every runtime error as a bottom, and even allow
functions to return the bottom explicitly. The latter is usually done
using the expression \code{undefined}, as in:

\src{snippet03}
This definition type checks because \code{undefined} evaluates to
bottom, which is a member of any type, including \code{Bool}. You can
even write:

\src{snippet04}
(without the \code{x}) because the bottom is also a member of the type
\code{Bool -> Bool}.

Functions that may return bottom are called partial, as opposed to total
functions, which return valid results for every possible argument.

Because of the bottom, you'll see the category of Haskell types and
functions referred to as $\Hask$ rather than $\Set$. From
the theoretical point of view, this is the source of never-ending
complications, so at this point I will use my butcher's knife and
terminate this line of reasoning. From the pragmatic point of view, it's
okay to ignore non-terminating functions and bottoms, and treat
$\Hask$ as bona fide $\Set$.\footnote{Nils Anders Danielsson,
John Hughes, Patrik Jansson, Jeremy Gibbons, \href{http://www.cs.ox.ac.uk/jeremy.gibbons/publications/fast+loose.pdf}{
Fast and Loose Reasoning is Morally Correct}. This paper provides justification for ignoring bottoms in most contexts.}

\section{Why Do We Need a Mathematical Model?}

As a programmer you are intimately familiar with the syntax and grammar
of your programming language. These aspects of the language are usually
described using formal notation at the very beginning of the language
spec. But the meaning, or semantics, of the language is much harder to
describe; it takes many more pages, is rarely formal enough, and almost
never complete. Hence the never ending discussions among language
lawyers, and a whole cottage industry of books dedicated to the exegesis
of the finer points of language standards.

There are formal tools for describing the semantics of a language but,
because of their complexity, they are mostly used with simplified
academic languages, not real-life programming behemoths. One such tool
called \newterm{operational semantics} describes the mechanics of program
execution. It defines a formalized idealized interpreter. The semantics
of industrial languages, such as C++, is usually described using
informal operational reasoning, often in terms of an ``abstract
machine.''

The problem is that it's very hard to prove things about programs using
operational semantics. To show a property of a program you essentially
have to ``run it'' through the idealized interpreter.

It doesn't matter that programmers never perform formal proofs of
correctness. We always ``think'' that we write correct programs. Nobody
sits at the keyboard saying, ``Oh, I'll just throw a few lines of code
and see what happens.'' We think that the code we write will perform
certain actions that will produce desired results. We are usually quite
surprised when it doesn't. That means we do reason about programs we
write, and we usually do it by running an interpreter in our heads. It's
just really hard to keep track of all the variables. Computers are good
at running programs --- humans are not! If we were, we wouldn't need
computers.

But there is an alternative. It's called \newterm{denotational semantics}
and it's based on math. In denotational semantics every programming
construct is given its mathematical interpretation. With that, if you
want to prove a property of a program, you just prove a mathematical
theorem. You might think that theorem proving is hard, but the fact is
that we humans have been building up mathematical methods for thousands
of years, so there is a wealth of accumulated knowledge to tap into.
Also, as compared to the kind of theorems that professional
mathematicians prove, the problems that we encounter in programming are
usually quite simple, if not trivial.

Consider the definition of a factorial function in Haskell, which is a
language quite amenable to denotational semantics:

\src{snippet05}
The expression \code{{[}1..n{]}} is a list of integers from \code{1} to \code{n}.
The function \code{product} multiplies all elements of a list. That's
just like a definition of factorial taken from a math text. Compare this
with C:

\begin{snip}{c}
int fact(int n) {
    int i;
    int result = 1;
    for (i = 2; i <= n; ++i)
        result *= i;
    return result;
}
\end{snip}
Need I say more?

Okay, I'll be the first to admit that this was a cheap shot! A factorial
function has an obvious mathematical denotation. An astute reader might
ask: What's the mathematical model for reading a character from the
keyboard or sending a packet across the network? For the longest time
that would have been an awkward question leading to a rather convoluted
explanation. It seemed like denotational semantics wasn't the best fit
for a considerable number of important tasks that were essential for
writing useful programs, and which could be easily tackled by
operational semantics. The breakthrough came from category theory.
Eugenio Moggi discovered that computational effect can be mapped to
monads. This turned out to be an important observation that not only
gave denotational semantics a new lease on life and made pure functional
programs more usable, but also shed new light on traditional
programming. I'll talk about monads later, when we develop more
categorical tools.

One of the important advantages of having a mathematical model for
programming is that it's possible to perform formal proofs of
correctness of software. This might not seem so important when you're
writing consumer software, but there are areas of programming where the
price of failure may be exorbitant, or where human life is at stake. But
even when writing web applications for the health system, you may
appreciate the thought that functions and algorithms from the Haskell
standard library come with proofs of correctness.

\section{Pure and Dirty Functions}

The things we call functions in C++ or any other imperative language,
are not the same things mathematicians call functions. A mathematical
function is just a mapping of values to values.

We can implement a mathematical function in a programming language: Such
a function, given an input value will calculate the output value. A
function to produce a square of a number will probably multiply the
input value by itself. It will do it every time it's called, and it's
guaranteed to produce the same output every time it's called with the
same input. The square of a number doesn't change with the phases of the
Moon.

Also, calculating the square of a number should not have a side effect
of dispensing a tasty treat for your dog. A ``function'' that does that
cannot be easily modelled as a mathematical function.

In programming languages, functions that always produce the same result
given the same input and have no side effects are called \newterm{pure
functions}. In a pure functional language like Haskell all functions are
pure. Because of that, it's easier to give these languages denotational
semantics and model them using category theory. As for other languages,
it's always possible to restrict yourself to a pure subset, or reason
about side effects separately. Later we'll see how monads let us model
all kinds of effects using only pure functions. So we really don't lose
anything by restricting ourselves to mathematical functions.

\section{Examples of Types}

Once you realize that types are sets, you can think of some rather
exotic types. For instance, what's the type corresponding to an empty
set? No, it's not C++ \code{void}, although this type \emph{is} called
\code{Void} in Haskell. It's a type that's not inhabited by any
values. You can define a function that takes \code{Void}, but you can
never call it. To call it, you would have to provide a value of the type
\code{Void}, and there just aren't any. As for what this function can
return, there are no restrictions whatsoever. It can return any type
(although it never will, because it can't be called). In other words
it's a function that's polymorphic in the return type. Haskellers have a
name for it:

\src{snippet06}
(Remember, \code{a} is a type variable that can stand for any type.)
The name is not coincidental. There is deeper interpretation of types
and functions in terms of logic called the Curry-Howard isomorphism. The
type \code{Void} represents falsity, and the type of the function
\code{absurd} corresponds to the statement that from falsity follows
anything, as in the Latin adage ``ex falso sequitur quodlibet.''

Next is the type that corresponds to a singleton set. It's a type that
has only one possible value. This value just ``is.'' You might not
immediately recognize it as such, but that is the C++ \code{void}.
Think of functions from and to this type. A function from \code{void}
can always be called. If it's a pure function, it will always return the
same result. Here's an example of such a function:

\begin{snip}{c}
int f44() { return 44; }
\end{snip}
You might think of this function as taking ``nothing'', but as we've
just seen, a function that takes ``nothing'' can never be called because
there is no value representing ``nothing.'' So what does this function
take? Conceptually, it takes a dummy value of which there is only one
instance ever, so we don't have to mention it explicitly. In Haskell,
however, there is a symbol for this value: an empty pair of parentheses,
\code{()}. So, by a funny coincidence (or is it a coincidence?), the
call to a function of void looks the same in C++ and in Haskell. Also,
because of the Haskell's love of terseness, the same symbol \code{()}
is used for the type, the constructor, and the only value corresponding
to a singleton set. So here's this function in Haskell:

\src{snippet07}
The first line declares that \code{f44} takes the type \code{()},
pronounced ``unit,'' into the type \code{Integer}. The second line
defines \code{f44} by pattern matching the only constructor for unit,
namely \code{()}, and producing the number 44. You call this function
by providing the unit value \code{()}:

\begin{snip}{c}
f44 ()
\end{snip}
Notice that every function of unit is equivalent to picking a single
element from the target type (here, picking the \code{Integer} 44). In
fact you could think of \code{f44} as a different representation for
the number 44. This is an example of how we can replace explicit mention
of elements of a set by talking about functions (arrows) instead.
Functions from unit to any type $A$ are in one-to-one correspondence with
the elements of that set $A$.

What about functions with the \code{void} return type, or, in Haskell,
with the unit return type? In C++ such functions are used for side
effects, but we know that these are not real functions in the
mathematical sense of the word. A pure function that returns unit does
nothing: it discards its argument.

Mathematically, a function from a set $A$ to a singleton set maps every
element of $A$ to the single element of that singleton set. For every $A$
there is exactly one such function. Here's this function for
\code{Integer}:

\src{snippet08}
You give it any integer, and it gives you back a unit. In the spirit of
terseness, Haskell lets you use the wildcard pattern, the underscore,
for an argument that is discarded. This way you don't have to invent a
name for it. So the above can be rewritten as:

\src{snippet09}
Notice that the implementation of this function not only doesn't depend
on the value passed to it, but it doesn't even depend on the type of the
argument.

Functions that can be implemented with the same formula for any type are
called parametrically polymorphic. You can implement a whole family of
such functions with one equation using a type parameter instead of a
concrete type. What should we call a polymorphic function from any type
to unit type? Of course we'll call it \code{unit}:

\src{snippet10}
In C++ you would write this function as:

\begin{snip}{cpp}
template<class T>
void unit(T) {}
\end{snip}
Next in the typology of types is a two-element set. In C++ it's called
\code{bool} and in Haskell, predictably, \code{Bool}. The difference
is that in C++ \code{bool} is a built-in type, whereas in Haskell it
can be defined as follows:

\src{snippet11}
(The way to read this definition is that \code{Bool} is either
\code{True} or \code{False}.) In principle, one should also be able
to define a Boolean type in C++ as an enumeration:

\begin{snip}{cpp}
enum bool { 
    true,
    false
};
\end{snip}
but C++ \code{enum} is secretly an integer. The C++11
``\code{enum class}'' could have been used instead, but then you
would have to qualify its values with the class name, as in
\code{bool::true} and \code{bool::false}, not to mention having to
include the appropriate header in every file that uses it.

Pure functions from \code{Bool} just pick two values from the target
type, one corresponding to \code{True} and another to \code{False}.

Functions to \code{Bool} are called \newterm{predicates}. For instance,
the Haskell library \code{Data.Char} is full of predicates like
\code{isAlpha} or \code{isDigit}. In C++ there is a similar library
\code{} that defines, among others, \code{isalpha} and
\code{isdigit}, but these return an \code{int} rather than a
Boolean. The actual predicates are defined in \code{std::ctype} and
have the form \code{ctype::is(alpha, c)}, \code{ctype::is(digit, c)}, etc.

\section{Challenges}

\begin{enumerate}
\tightlist
\item
  Define a higher-order function (or a function object) \code{memoize}
  in your favorite language. This function takes a pure function
  \code{f} as an argument and returns a function that behaves almost
  exactly like \code{f}, except that it only calls the original
  function once for every argument, stores the result internally, and
  subsequently returns this stored result every time it's called with
  the same argument. You can tell the memoized function from the
  original by watching its performance. For instance, try to memoize a
  function that takes a long time to evaluate. You'll have to wait for
  the result the first time you call it, but on subsequent calls, with
  the same argument, you should get the result immediately.
\item
  Try to memoize a function from your standard library that you normally
  use to produce random numbers. Does it work?
\item
  Most random number generators can be initialized with a seed.
  Implement a function that takes a seed, calls the random number
  generator with that seed, and returns the result. Memoize that
  function. Does it work?
\item
  Which of these C++ functions are pure? Try to memoize them and observe
  what happens when you call them multiple times: memoized and not.

  \begin{enumerate}
  \tightlist
  \item
    The factorial function from the example in the text.
  \item
\begin{minted}{cpp}
std::getchar()
\end{minted}
  \item
\begin{minted}{cpp}
bool f() { 
    std::cout << "Hello!" << std::endl;
    return true;
}
\end{minted}
  \item
\begin{minted}{cpp}
int f(int x) {
    static int y = 0;
    y += x;
    return y;
}
\end{minted}
  \end{enumerate}
\item
  How many different functions are there from \code{Bool} to
  \code{Bool}? Can you implement them all?
\item
  Draw a picture of a category whose only objects are the types
  \code{Void}, \code{()} (unit), and \code{Bool}; with arrows
  corresponding to all possible functions between these types. Label the
  arrows with the names of the functions.
\end{enumerate}