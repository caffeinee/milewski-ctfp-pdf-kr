% !TEX root = ../../ctfp-print.tex

\begin{quote}
한동안 나는 프로그래머들을 대상으로 \trCategoryTheory\을 설명하는 책을 쓰는 것에 대한 생각에 사로잡혔다. 
그러니까, 컴퓨터 과학자가 아니라 프로그래머들, 혹은 엔지니어들을 대상으로 하는 책 말이다.
이 말이 조금 미친 것처럼 들릴 수 있다는 것을 안다. 꽤 겁이 나기도 한다.
두 분야에서 모두 일해 본 적이 있기 때문에 과학과 공학 사이에는 큰 차이가 있다는 사실을 부정할 수 없다.
하지만, 나는 항상 이것을 설명하고 싶다는 강한 충동을 느낀다. 나는 간결한 설명의 대가인 리처드 파인만에 대해 큰 존경심을 품고 있다.
물론 파인만처럼 해낼 수는 없겠지만 내 나름대로 최선을 다해서 설명해보려고 한다.
이 서문을 통해 토의가 시작되고 피드백이 돌아왔으면 좋겠다.
또 독자에게 \trCategoryTheory\을 배우고자 하는 의욕을 심어줄 수 있었으면 한다.
\footnote{
관중들 앞에서 이 내용을 가르치는 영상도 \href{https://goo.gl/GT2UWU}{https://goo.gl/GT2UWU}(혹은 ``bartosz milewski category theory''를 youtube에서 검색)에서 볼 수 있다.}
\end{quote}

어쩌면 이 글을 읽는 사람들 중에는 왜 이런 극도로 추상적인 수학 공부를 하는데 내 여가 시간을 투자해야만 하냐고 반론을 하는 사람이 있을 지도 모른다.
그래서 다음 몇 문단을 통해 이 책은 당신을 위해 쓴 책이며 충분히 읽을 가치가 있을 것이라는 사실을 납득시켜 보겠다.

내 낙관론은 몇 가지 관찰에 기반한다. 첫째로, \trCategoryTheory\는 굉장히 유용한 프로그래밍 아이디어의 보물 창고다.
하스켈 프로그래머들은 이 보물 창고를 오랜 시간동안 두드려 왔으며, 이 아이디어들은 천천히 다른 언어들로 침투해가고 있다.
하지만 그 과정은 너무나 느리다. 우리는 이를 가속시킬 필요가 있다.

둘째로, 세상에는 굉장히 다양한 종류의 수학이 있으며 사람마다 흥미를 가지는 분야는 서로 다르다.
대수학 혹은 미적분학에 거부감을 가지고 있을 수도 있지만, 그게 곧 \trCategoryTheory\를 즐길 수 없다는 의미는 아니다.
나는 \trCategoryTheory\는 특히 프로그래머들의 마음에 잘 맞는다라고까지 이야기하고 싶다. 왜냐하면 \trCategoryTheory\는 구체적인 대상들을 다루기보단 구조를 다루기 때문이다.
\trCategoryTheory\는 프로그램을 합성 가능하게(composable) 만드는 종류의 구조에 대해 다룬다.

\trComposition\은 \trCategoryTheory의 근간이다. 이 것은 \trCategory의 정의 그 자체의 일부분이기도 하다.
또한 나는 \trComposition이 프로그래밍의 정수라고 강력하게 주장하고 싶다.
우리는 먼 옛날 몇몇 위대한 엔지니어들이 서브루틴(subroutine)의 개념을 찾아낸 뒤로부터 계속해서 무언가들을 합성해오고 있다.
예전 구조적 프로그래밍의 원칙은 코드 블록을 합성할 수 있게 만듦으로써 프로그래밍에 혁명을 가져왔다.
그리고 그 이후 모든 객체들을 합성하는 객체 지향 프로그래밍이 나타났다.
함수형 프로그래밍은 함수와 대수적 자료 구조들을 합성하는 것 뿐만 아니라, 다른 프로그래밍 패러다임에서는 사실상 불가능한 동시성(concurrency)과 같은 것들까지도 합성한다.

셋째로, 나는 수학을 프로그래머들의 입맛에 맞게 조리할 수 있는 비밀의 무기를 갖고 있다.
만약 당신이 전문적인 수학자라면, 자신이 한 가정이 전부 올바른지 보고, 모든 문장이 적절한지 확인하고, 모든 증명을 엄밀하게 구성하기 위해 굉장히 신경을 쓸 것이다.
이런 일련의 과정이 수학 논문과 서적을 외부인들이 읽기 극도로 힘들게 만든다. 나는 물리학 전공자인데, 물리학에서는 정형화되지 않은 추론으로 굉장한 진보를 이뤄낸다.
수학자들은 위대한 물리학자 P. A. M. 디랙이 몇몇 미분 방정식을 풀기 위해 즉석으로 만들어낸 디랙 델타 함수를 보고 이를 비웃음거리로 삼았다.
하지만, 디랙의 통찰을 공식화한 완벽하게 새로운 미적분학의 한 갈래인 분포 이론(distribution theory)을 발견하곤 곧 웃음을 멈췄다.

물론 엄밀하지 않게 대략적으로 이야기하는 것은 뻔뻔하게도 완전 잘못된 이야기를 하게 만들 위험성을 내포하고 있다.
그래서 나는 이 책에 약식으로 기술한 요소들 뒤에 엄밀한 수학적 토대가 있음을 확인할 수 있게 하려 노력할 것이다.
나는 다 닳아 헤진 손더스 매클레인(Saunders Mac Lane)의 \emph{Category Theory for the Working Mathematician} 한 권을 내 침실 탁자 위에 올려두고 있다.

이것이 \emph{프로그래머를 위한} \trCategoryTheory이기 때문에 나는 모든 중요한 개념들을 컴퓨터 코드를 이용해 나타낼 것이다.
당신은 아마 함수형 언어가 더 유명한 명령형 언어들보다 수학에 가깝다는 사실을 인지하고 있을 것이다. 함수형 언어는 더욱 강력한 추상적인 힘을 가져다 준다.
그래서 자연스럽게 "\trCategoryTheory\로 인한 이득을 사용할 수 있게 되려면 너는 반드시 먼저 하스켈을 배워야 해"라고 말하고픈 유혹에 빠지게 된다.
하지만 이는 \trCategoryTheory\가 함수형 프로그래밍 외에는 전혀 쓸 곳이 없다는 의미를 내포하게 되는데, 이것은 전혀 사실이 아니다. 따라서 나는 많은 양의 C++ 예제도 제공할 것이다.
물론, 일부 흉측한 구문들을 극복해야할 것이며, 패턴들은 그 배경의 장황함 때문에 두드러져 보이지 않을 것이고, 고수준의 추상화 대신에 복사 붙여넣기를 해야만 할 것이다.
하지만 어차피 그건 C++ 프로그래머라면 늘상 하는 일일 뿐이다.

하지만 하스켈에 관한 고려를 아예 내려놓지는 않아야 한다. 하스켈 프로그래머가 될 필요는 없지만, C++에 구현된 아이디어를 입증하고 스케치하기 위한 언어로서 하스켈이 필요하다.
내가 하스켈을 시작하게 된 이유도 이와 같다. 나는 하스켈의 간단명료한 구문과 강력한 타입 시스템이 C++의 템플릿, 자료 구조, 알고리즘을 이해하고 구현하는데 굉장히 큰 도움이 된다고 본다.
그러나 독자들이 이미 하스켈을 알고 있다고 가정할 수는 없기 때문에, 차차 진행하면서 모든 것을 천천히 소개하고 설명할 것이다.

당신이 이미 숙련된 프로그래머라면, 스스로에게 이런 질문을 해보았을 것이다. 
"나는 오랫동안 \trCategoryTheory에 대한 걱정이나 함수형 기법에 대한 걱정 없이 잘 코딩해 왔는데, 뭔가 달라진 게 있어?"
물론 당신은 이미 함수형 언어의 기능들이 명령형 언어에 침투해 들어오고 있는 꾸준한 흐름을 이미 느끼고 있을 것이다.
심지어 객체 지향 프로그래밍의 마지막 보루로 느껴졌던 자바마저도 람다가 들어갔다.
C++은 최근 몇년에 한번씩 새로운 표준이 나올 만큼 제정신이 아닌 속도로 발전하고 있다. 변화하는 세상을 따라잡기 위해서 말이다.
이런 모든 활동들은 파괴적인 변화, 혹은 물리학자로서 말하자면 상전이(phase transition)를 위한 준비에 속한다고 볼 수 있다.
물을 계속 데우다 보면 언젠간 끓기 시작할 것이다. 우리는 점점 뜨거워지는 물 속에서 계속 헤엄을 칠지, 혹은 다른 대안을 찾든지 선택의 기로에 놓인 개구리의 위치에 있는 것이다.

\begin{figure}[H]
\centering
\includegraphics[width=0.5\textwidth]{images/img_1299.jpg}
\end{figure}

\noindent
이 거대한 변화를 이끌고 있는 힘 중 하나는 멀티코어 혁명이다.
기존의 주류 프로그래밍 패러다임이던 객체 지향은 동시성과 병렬성의 영역에서는 어느 것도 가져다 주지 못 한다.
대신 위험하고 버그가 많은 디자인만 야기할 뿐이다.
객체 지향의 기본적인 전제인 정보 은닉(Data hiding)은 공유(sharing), 그리고 조작(mutation)과 결합될 경우 그대로 데이터 레이스(data race)를 만들어 버린다.
뮤텍스와 그 뮤텍스가 보호할 데이터를 결합하는 것은 좋은 생각이지만 안타깝게도 락(lock)은 합성할 수 없고, 락을 숨기는 것은 데드락이 발생할 위험이 커지며 더욱 디버깅을 어렵게 만든다.

그러나 병렬성을 떼놓고 보더라도, 점점 증가하는 소프트웨어 시스템의 복잡성은 명령형 패러다임의 확장성의 한계를 테스트하고 있다.
간단히 말해, \trSideEffect(side effects)\은 점점 손 쓸 수 없는 범위로 가고 있다. 물론, \trSideEffect\을 가진 함수는 때때로 편리하고 작성하기도 쉽다.
이들이 끼치는 영향은 원칙적으로 그 함수의 이름과 주석에 포함될 수 있다.
SetPassword 혹은 WriteFile과 같은 이름의 함수는 명백하게 어떤 상태를 변경하고, \trSideEffect\을 만들어낼 것이며 우리는 이것을 다루는 데에 익숙하다.
그러나 \trSideEffect\을 가진 함수를 또 다른 \trSideEffect\을 가진 함수와 합성하고, 다시 또 다른 함수와 합성하고, 반복하다보면 금세 불편해진다. 
\trSideEffect 자체가 선천적으로 나쁘다는 것이 아니다. 이것은 그저 \trSideEffect\이 볼 수 없게 숨어 있는 것이 이를 더 큰 규모에서 관리하는 게 불가능하게 만든다는 사실일 뿐이다.
\trSideEffect\은 확장할 수 없고, 명령형 언어는 모두 \trSideEffect에 관한 것이다.

하드웨어의 변화와 소프트웨어 복잡성의 증가는 우리가 프로그래밍의 기반에 대해 다시 생각하게끔 강요하고 있다.
유럽의 고딕 대성당 건축가들처럼, 우리는 우리가 계속해서 써오던 도구들, 구조들에 대한 프로그래밍 기술을 계속해서 연마해 왔다.
\urlref{http://en.wikipedia.org/wiki/Beauvais_Cathedral}{보베 대성당}이라는 완성되지 못한 고딕 대성당이 프랑스에 있다.
이 성당은 한계에 대한 인간의 깊은 투쟁에 대한 목격자로서 서 있다.
이 성당은 이전의 모든 높이와 가벼움에 대한 기록을 깨부수려고 했으나, 여러 차례 붕괴 사고를 겪고 말았다.
철제 지지대와 목판 보강과 같은 이후의 즉석 조치들은 건물이 완전히 무너지지 않게끔 지키고 있지만, 여러 가지가 잘못되어 가고 있다는 것은 명확하다.
현대적인 관점에서, 대부분의 고딕 구조가 현대 재료 과학, 컴퓨터 모델링, 유한 요소 해석, 그리고 일반적인 수학과 물리학의 도움 없이 성공적으로 완공되었다는 사실은 거의 기적에 가깝다.
나는 후대 사람들이 우리가 복잡한 운영 체제, 웹 서버, 그리고 인터넷 인프라를 구축하는데 사용해온 프로그래밍 기술을 감탄의 시선으로 바라보기를 희망한다.
그리고, 노골적으로 말하자면, 그들은 그럴 수 밖에 없을 것이다. 왜냐하면 우리는 이 모든 것을 굉장히 조잡한 이론적 토대만 가지고 만들어오고 있기 때문이다.
다음 단계로 나아가기 위해서 우리는 우리의 이 토대를 고쳐야만 한다.

\begin{figure}
\centering
\includegraphics[totalheight=0.5\textheight]{images/beauvais_interior_supports.jpg}
\caption{보베 대성당을 붕괴로부터 보호하고 있는 즉석 조치들}
\end{figure}
