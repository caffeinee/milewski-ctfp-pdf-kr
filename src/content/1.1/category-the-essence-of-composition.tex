% !TEX root = ../../ctfp-print.tex

% 본 번역본을 검토하는 데에 있어서 다음과 같이 해 주시기를 희망합니다.
% 1. 복붙
% 2. preamble을 연다
% 3. \tr~~~~들을 전부 해당하는 한국말로 대체해서 읽어본다.

\lettrine[lhang=0.17]{카}{테고리}는 부끄러울 정도로 간단한 개념이다.
\trCategory\는 \newterm{\trObject}\와 그들 사이를 잇는 \newterm{\trArrow}\로 이루어져 있다. 
그래서 \trCategory\는 그림으로 나타내기 쉽다. \trObject\는 원이나 점으로, 
\trArrow\는\ldots{} 화살표로 그린다. (독자가 질리지 않게끔 
나는 가끔 \trObject 대신 돼지를, \trArrow 대신 폭죽을 가지고 그림을 그릴 것이다.)
그러나, \trCategory의 정수는 \emph{\trComposition}이다. 또는, 이렇게 말하는 걸 더 좋아한다면, \trComposition의 정수는 \trCategory이다.
\trArrow\는 \trComposition된다. 
즉 \trObject $A$에서 \trObject $B$로 가는 \trArrow\가 하나, \trObject $B$에서 \trObject $C$로 가는 \trArrow\가 하나 있다면, 
$A$ 에서 $C$로 가는 \trArrow\가(\trComposition\이) 있어야 한다.

\begin{figure}
\centering
\includegraphics[width=0.8\textwidth]{images/img_1330.jpg}
\caption{어떤 \trCategory에서 $A$에서 $B$로, $B$에서 $C$로 가는 화살표가 각각 있다면,
이 두 화살표의 \trComposition, 즉, $A$에서 $C$로 직접 가는 화살표가 있어야 한다.
이 도식은 \trIdentityMorphism(후술)\이 없어서 완전한 \trCategory\는 아니다.}
\end{figure}

\section{함수로써의 \trArrow}

이미 너무 뜬구름잡는 것 같은가? 절망하지 마라. 그럼 조금 실질적인 이야기를 해 보자.
\trArrow\를, 혹은 다른 이름으로 \newterm{\trMorphism}\을, 함수라고 생각해 보라.
당신은 $A$타입 \trArgument\를 받아서 $B$타입 객체를 리턴하는 함수 $f$를 가지고 있다.
또 당신은 $B$타입 \trArgument\를 받아 $C$타입 객체를 리턴하는 함수 $g$도 가지고 있다.
당신은 $f$의 리턴값을 $g$에 제공해서 $f$와 $g$를 \trComposition할 수 있다.
이렇게 해서 $A$를 받아서 $C$를 리턴하는 함수를 만들었다.

수학에서는 함수 이름 사이에 작은 원을 그려서 \trComposition\을 표시한다 ($g \circ f$). 
\trComposition 순서가 오른쪽에서 왼쪽이라는 점에 주의하라. 일부 독자에게는 이 순서가 헷갈릴 수 있다.
독자는 아래와 같은 유닉스 파이프 명령(pipe notation)에 친숙할 지도 모르겠다.

\begin{snip}{text}
lsof | grep Chrome
\end{snip}

또는, F\#의 셰브론(Chevron) \code{>>}에 친숙할 수도 있다. 이들은 왼쪽에서 오른쪽으로 진행한다.
그러나 수학의 함수들과 하스켈의 함수들은 오른쪽에서 왼쪽으로 \trComposition된다.
$g \circ f$를 ``g \emph{하기 전에} f.''라고 읽으면 좀 덜 헷갈린다.
%맙소사 이거 어쩌죠 아악%

C언어 코드를 써서 조금 더 명확하게 해 보자. 
우리는 \code{A}타입 \trArgument\를 받아서 \code{B}타입 값을 리턴하는 함수 \code{f}를 가지고 있다.

\begin{snip}{text}
B f(A a);
\end{snip}
이며

\begin{snip}{text}
C g(B b);
\end{snip}
면, 이 둘의 \trComposition\은 다음과 같다.

\begin{snip}{text}
C g_after_f(A a)
{
    return g(f(a));
}
\end{snip}
C언어에서도 오른쪽에서 왼쪽 순서의 \trComposition\을 볼 수 있다. (\code{g(f(a))})

C++ 표준 라이브러리에 함수 두 개를 받아서 그 \trComposition\을 리턴하는 템플릿이 있다 말할 수 있으면 참 좋겠으나, 그런 건 없다.
그러므로 한 번 하스켈을 써 보자. A에서 B로 가는 함수는 다음과 같이 선언한다.

\src{snippet01}
비슷하게, g를 다음과 같이 선언할 수 있다.

\src{snippet02}
이들의 \trComposition\은 아래와 같다.

\src{snippet03}
하스켈에서의 표현이 얼마나 간단한지 보고 나면, 이리도 쉬운 함수형 개념들을 C++에서 표현할 수 없다는 것이 조금 부끄럽게 느껴진다.
사실 하스켈은 유니코드를 지원하기 때문에, \trComposition\을 다음과 같이 쓸 수 있다.
% don't 'mathify' this block
\begin{snip}{text}
g ◦ f
\end{snip}

심지어 유니코드 더블 콜론($\ensuremath{\Colon}$)과 화살표도 쓸 수 있다.
% don't 'mathify' this block
\begin{snipv}
f \ensuremath{\Colon} A → B
\end{snipv}
%여기 말인데, 좋은 아이디어 있으시면 제시 부탁드립니다.%
자, 여기서 첫 번째 하스켈 수업이다. 
더블 콜론은 ``~~타입을 가지는''이라는 뜻이다.
두 타입 사이에 화살표를 넣어서 함수 타입을 만들 수 있다.
두 함수 사이에 온점( . )이나 유니코드 원(◦)을 넣어서 \trComposition할 수 있다.

\section{\trComposition의 성질}

\trCategory에 속한 \trComposition\이 만족해야 할 매우 중요한 성질이 두 가지 있다.

\begin{enumerate}
\item
\trComposition\은 \trAssociativity\을 만족한다. \trComposition 가능한 세 개의 \trMorphism $f$, $g$, $h$가 있을 때
(즉 양 끝의 \trObject\가 서로 잘 맞아떨어질 때), 이들의 \trComposition에는 괄호가 필요 없다.
수학 기호로는 다음과 같이 쓴다.
\[h \circ (g \circ f) = (h \circ g) \circ f = h \circ g \circ f\]
(의사) 하스켈에서는 다음과 같이 쓸 수 있다.

\src{snippet04}[b]
(``의사'' 하스켈이라고 한 것은, 하스켈에는 함수 사이의 동등(==) 연산이 정의되어있지 않지 때문이다.)

\trAssociativity\은 함수를 다룰 때는 당연해 보이나, 다른 카테고리에서는 꼭 그렇지만은 않을 수 있다.

\item
모든 \trObject $A$에 대해, \trComposition의 \trUnit\이 되는 \trArrow\가 존재한다.
이 \trArrow\는 $A$에서 자신으로 다시 돌아간다. 
\trComposition의 \trUnit임은, $A$에서 시작하거나 끝나는 \trArrow와 \trComposition한 결과가 그 각각의 화살표라는 의미이다.
\trObject $A$와 연관된 \trUnit \trArrow\는 $\idarrow[A]$ ($A$에의 \newterm{\trIdentity}) 라고 부른다.
수학 기호로는, $f$가 $A$를 받아 $B$를 반환한다면
\[f \circ \idarrow[A] = f\]
이며
\[\idarrow[B] \circ f = f\]
이다.
\end{enumerate}
함수와 관련된 작업을 할 때, \trIdentity\을 나타내는 \trArrow\는 그 \trArgument\를 그대로 리턴하는 항등함수로 구현된다.
이 구현은 모든 타입에 대해 같으므로, 보편 다형적(universally polymorphic)이다. 
C++에서는 템플릿으로 다음과 같이 정의할 수 있을 것이다.

\begin{snip}{cpp}
template<class T> T id(T x) { return x; }
\end{snip}
물론, C++에서는 \trArgument의 타입뿐만 아니라 어떻게 (값, 레퍼런스, const, 이동, 등등...) \trArgument\를 넘겨주는지도 고려해야 하므로,
실제로는 이렇게 간단하지 않다.

하스켈에서는 항등함수가 표준 라이브러리(Prelude)에 포함되어 있다. 
그 선언과 정의는 다음과 같다.

\src{snippet05}
이렇듯, 하스켈에서는 다형적 함수 다루기는 누워서 떡 먹기다.
선언에서는 자료형을 타입 변수(type variable)로 대체하면 끝이다.
Integer, Char과 같은 구체적인 자료형의 이름은 항상 대문자로, 타입 변수의 이름은 항상 소문자로 시작함을 기억하자.
즉, 여기서 \code{a}는 모든 타입을 나타낼 수 있다.

하스켈에서 함수 정의는 함수의 이름, 그리고 이를 뒤따르는 \trFormalParameter\로 이루어진다.
위 예시에서 \trFormalParameter\는 \code{x} 단 하나다.
함수의 \trFunctionBody(body)\는 등호(=) 뒤에 붙는다.
하스켈 새내기들은 간혹 이러한 간결함에 놀라기도 하지만, 그게 말이 된다는 점은 바로 알 수 있을 것이다.
함수형 프로그래밍에서 함수 정의와 호출은 밥이나 빵만큼 중요하므로, 관련 문법을 최소화한 것이다.
그래서 \trArgument 목록 주위에 소괄호가 없을 뿐만 아니라, \trArgument 사이에 반점(,)도 쓰지 않는다.
(나중에 다\trArgument 함수를 정의할 때 이 사실을 알 수 있다.)

함수의 \trFunctionBody\는 항상 \trExpression(expression)이다. 함수에서는 \trStatement(statement)\는 없다. 
함수의 결과값은 \trExpression이며 여기서는 \code{x}다.

이로써 두 번째 하스켈 강의를 끝마쳤다.
%여기도...%

항등 조건은 의사하스켈로 다음과 같이 쓸 수 있다.

\src{snippet06}
문득 이러한 질문이 떠올랐을 수 있다. 대체 왜 아무 일도 안 하는 항등함수에 신경을 쓰는가?
그럼 숫자 0에는 신경을 왜 쓰는가? 0은 무(無)의 상징이다. 고대 로마인들은 0 없는 숫자 체계를 가지고 있었지만
우수한 도로와 수도 시설을 만들어냈으며, 이들 중 일부는 오늘날까지도 남아있다.

0이나 $id$같은 중립적인 값들은 \trSymbolic 변수를 다룰 때에 유용하다.
그래서 로마인들은 대수학을 잘 못 했고, 반면 0을 알고 있던 아랍인들과 페르시아인들은 대수학을 잘 했다.
어쨌든, 항등함수는 \trHigherOrderFunction의 \trArgument, 혹은 리턴값으로 쓰기 유용하다. 
그리고 \trHigherOrderFunction\가 함수의 \trSymbolic적인 조작을 가능케 한다.
즉, 함수를 가지고 전개하는 대수학이 가능해지는 것이다.

요약하면, \trCategory\는 \trObject\와 \trArrow(\trMorphism)\로 이루어져 있다.
\trArrow\는 \trComposition될 수 있으며, 이 \trComposition\은 \trAssociativity\을 만족한다.
그리고, 모든 \trObject에는 \trComposition의 \trUnit인 \trIdentity \trArrow\가 있다.

\section{\trComposition\은 프로그래밍의 정수다}

함수형 프로그래머들은 특이한 방식으로 문제에 접근한다. 그들은 무척이나 철학자 같은 질문을 한다.
이를테면, 인터렉티브 프로그램을 만든다면, 그들은 먼저 인터렉션이 무엇인지를 묻는다.
콘웨이의 생명 게임을 구현한다면, 아마 그들은 인생이란 무엇인지 고민해볼 것이다.
이러한 관례를 따라 필자는 다음과 같은 질문을 던지려 한다. 프로그래밍이란 무엇인가?
가장 기본적인 수준에서 얘기하자면, 프로그래밍이란 컴퓨터에게 명령을 내리는 것이다.
``메모리 주소 x에 있는 내용을 EAX 레지스터의 내용에 더하시오.'' 
그러나 어셈블리어로 프로그램을 짤 때조차도, 우리가 컴퓨터에게 주는 명령은 무언가 더 의미있는 다른 것의 표현에 지나지 않는다.
우리는 꽤 큰 규모의 문제를 해결하기 위해서 컴퓨터를 사용한다 (소규모 문제라면 컴퓨터를 사용해 풀지 않을 것이다.)
그런데 문제는 또 어떻게 해결하는가? 우리는 큰 문제들을 작은 문제들로 쪼갠다. 
이 작은 문제들이 아직도 너무 크다면, 계속 더 잘게 쪼갠다.
마지막으로, 이렇게 나온 모든 작은 문제들을 해결할 수 있는 코드를 작성한다.
이 다음이 프로그래밍의 정수다.
우리는 코드 조각조각을 잘 \trComposition해서, 큰 문제의 해답을 만든다. 
만약 이렇게 코드를 다시 합지는 게 불가능했다면, 쪼개는 것은 애초에 말이 되지 않는다.

이러한 계층적인 분해와 조립은 컴퓨터가 우리에게 강제한 것이 아니다.
대신 인간 정신의 한계를 반영한 것일 뿐이다. 우리의 뇌는 한 번에 단 몇 개의 개념만을 다룰 수 있다.
심리학 분야에서 가장 많이 인용된 논문인 
\urlref{http://en.wikipedia.org/wiki/The_Magical_Number_Seven,_Plus_or_Minus_Two}{The
Magical Number Seven, Plus or Minus Two}
에서 저자는 우리가 한 번에 $7 \pm 2$ ``조각''의 정보만을 기억하고 있을 수 있다고 추측했다.
물론 단기기억능력에 관한 학설이 시간에 따라 바뀌어가지만, 어쨌든 단기기억력에 한계가 있음은 명확하다.
요는, 우리는 너무 많은 객체나, 꼬일 대로 꼬인 스파게티 코드에 대처할 수 없다는 점이다.
구조라는 것이 필요한 이유는 구조 잘 잡힌 코드를 보면 기분이 좋기 때문이 아니라,
단지 우리의 뇌가 무질서한 코드를 잘 처리해낼 수 없어서다.
가끔 우리는 어떤 코드를 보고 예쁘거나 우아하다고 표현하지만, 이 말은 사실
사람의 한정된 뇌 자원으로 그 코드를 이해하기가 쉽다고 이야기하는 것이다.
우아한 코드는 우리가 소화하기 쉬운 크기와 갯수의 조각들을 만들어내는 코드다.

그래서 \trComposition하기에 알맞은 프로그램 조각들이란 무엇인가?
이 조각들의 ``표면적''이 ``부피'' 보다 느리게 증가해야 한다.
(나는 이 비유가 좋다. 기하학적인 물체의 표면적은 길이 제곱에 비례해 증가한다.
길이 세제곱에 비례하는 부피보다 표면적이 느리게 증가한다는 사실은 직감적으로 받아들이기 쉽다.)
``표면적''이란 조각을 \trComposition해내기 위해 필요한 정보를 말한다.
``부피''란 그 조각을 구현해내기 위해 필요한 정보를 말한다.
어떤 조각이 완성되면 그 구현에 대해서는 잊어버리고,
대신 다른 조각과 어떻게 상호작용하는지에 대해서만 집중하자는 이야기다.
객체지향 프로그래밍에서는, ``표면적''은 클래스 선언문, 추상 인터페이스 등이다.
함수형 프로그래밍에서는 함수의 선언이 ``표면적''에 해당한다. (조금 너무 간략화시켜서 이야기했지만, 대강 그렇다.)

\trCategoryTheory\는, \trObject 안을 들여다 보는 행위를 적극적으로 막는다는 면에서 극단적이다.
\trCategoryTheory 상에서 \trObject\는 추상적이고 모호한 객체이다.
이 \trObject\가 다른 \trObject\와 어떻게 관련되는지, 즉 \trArrow\로 어떤 식으로 연결되어 있는지만을 알 수 있다.
인터넷 검색 엔진들이 (``반칙''을 쓰지 않으면) 이런 식으로 작동하는데, 
사이트에서 들어가고 나가는 링크들을 분석해서 웹 사이트들에 순위를 매긴다.
객체지향 프로그래밍에서 이상적인 객체는 추상적 인터페이스만을 통해 볼 수 있고(``표면적''만 있고 ``부피''는 없다),
여기서 \trArrow의 역할을 하는 것은 메소드이다. 
만약 어떤 \trObject\를 다른 \trObject\와 \trComposition할 때 \trObject의 구현을 들여다봐야 한다면
이미 그 프로그래밍 패러다임의 장점은 없는 것이나 다름 없다.

\section{연습문제}

\begin{enumerate}
\tightlist
\item
  당신이 가장 좋아하는 프로그래밍 언어로 항등함수를 구현하라.
  (만약 그 언어가 하스켈이라면, 두 번째로 좋아하는 프로그래밍 언어를 사용하라.)
\item
  당신이 가장 좋아하는 프로그래밍 언어로 두 함수의 \trComposition\을 리턴하는 함수를 작성하라.
  \trArgument\로 두 함수가 주어지고, 이 두 함수의 \trComposition\이 리턴되어야 한다.  
\item
  바로 위 문제에서 짠 함수가 항등성을 만족하는지 테스트하려 시도하는 코드를 짜라.
\item
  월드 와이드 웹을 \trCategory\라 볼 수 있는가? 그렇다면 하이퍼링크는 \trMorphism인가?
\item
  사람을 \trObject, 페이스북 친구 관계를 \trMorphism\이라 본다면
  페이스북을 \trCategory\라 볼 수 있는가?
\item
  유향그래프는 어떤 조건이 주어지면 \trCategory인가?
\end{enumerate}
